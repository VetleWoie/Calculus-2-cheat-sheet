\documentclass[8pt,a4paper,twocolumn,twoside]{article}
\usepackage[a4paper,total={20cm, 28.5cm},headsep=0.3cm]{geometry}
\usepackage[utf8]{inputenc}
\usepackage{sectsty}
\usepackage{titlesec}
\usepackage{amsfonts}
\usepackage{enumitem}
\usepackage{amsmath}

\titleformat{\subsection}{\small\bfseries}{\thesection}{1em}{}
\titleformat{\subsubsection}{\footnotesize\bfseries}{\thesection}{1em}{}
\sectionfont{\fontsize{10}{8}\selectfont}

\titlespacing*{\section}{0pt}{0.2cm}{0.1cm}
\titlespacing*{\subsection}{0pt}{0.2cm}{0.1cm}
\titlespacing*{\subsubsection}{0pt}{0.2cm}{0.1cm}

\setlength{\columnseprule}{1pt}

\def\abs#1{\lvert #1 \rvert}
\def\real{\mathbb{R}}
\def\suminfty#1#2{\sum_{n=#1}^\infty #2}

\begin{document}
    \section*{Uegentige Integraler:}
        \subsection*{Sammenligningskriteriet:}
        La $f,g : [a:\infty)\rightarrow \real$ , kontinuerlig og positiv. Anta
        $f(x)\geq(x)$ for alle x:
        \begin{enumerate}[topsep = 0pt,partopsep = 0pt, itemsep = 0cm]
            \item Hvis $\int_a^\infty f(x) dx$ Konvergerer $\Rightarrow$ $\int_a^\infty g(x) dx$ Konvergerer
            \item Hvis $\int_a^\infty g(x) dx$ Divergerer $\Rightarrow$ $\int_a^\infty f(x) dx$ Divergerer
        \end{enumerate}
        \subsection*{Grensesammenligningskriteriet:}
        La $f,g : [a:\infty)\rightarrow \mathbb{R}$ , kontinuerlig og positiv.
        \begin{enumerate}[topsep = 0pt,partopsep = 0pt, itemsep = 0cm]
            \item $\int_a^\infty f(x)dx$ Konvergerer og $\lim_{x\to\infty}\frac{g(x)}{f(x)}<\infty \Rightarrow \int_a^\infty g(x)$ Konvergerer 
            \item $\int_a^\infty f(x)dx$ Divergerer og $\lim_{x\to\infty}\frac{g(x)}{f(x)}>0 \Rightarrow \int_a^\infty g(x)$ \\Divergerer
        \end{enumerate}
        \subsection*{Viktige Integraler:}
        $\int_0^1 \frac{dx}{x^p}$ Konvergerer for $p<1$, divergerer for $p\geq1$\\
        $\int_1^\infty\frac{dx}{x^p}$Konvergerer for $p>1$,divergerer for $p\leq1$\\
    
    \section*{Taylorpolynom:}
        \subsection*{Taylors formel med restledd:}
        Anta f og den n+1 første deriverte er kont på $[a,b]$:\\
        $f(b) = T_n f(b)+\frac{1}{n!}\int_a^b f^{n+1}(t)(b-t)^n dt$
        \subsection*{Lagranges restleddformel}
        Anta f of dens n+1 første deriverte er kont på $[a,b]$\\
        $R_nf(x) = \frac{f^{n+1}(c)}{(n+1)!}(x-a)^{n+1}$
    \section*{Funksjonsfølger:}
        \subsection*{Punktvis og uniform konvergens:}
            \subsubsection*{Definisjon av punktvis konvergens:}
            La $\{f_n\}$ være en følge som er definert på en mengde A, og la $f$ være en funksjon definert på 
            samme mengde A. ${f_n}$ Konvergerer punktvis mot $f$ på A, Hvis:
            $\lim_{n\to\infty}f_n(x)=f(x)$ for alle x i A
            \subsubsection*{Definisjon av avstand mellom to funksjoner over A:}
            $f$ og $g$ er definert på samme mengde A. avstanden blir da:
            $d_A(f,g) = sup\{\abs{f(x)-g(x):x\in A}\}$
            \subsubsection*{Definisjon av uniform kovergens:}
            En funksjonsfølge $\{f_n\}$, definert på A, konvergerer uniformt mot $f$(Også definert på A) hvis:
            $\lim_{n\to\infty}d_A(f,f_n) = 0$
            \subsubsection*{Dinis teorem:}
            Anta at $\{f_n\}$ er en voksende følge av kont. funksjoner som konvergerer punktvis mot en kont.
            funksjon $f$ på et lukket, begrenset intervall $[a,b]$. Da konvergerer $\{f_n\}$ uniformt mot $f$ på $[a,b]$
        \subsection*{Integrasjon og derivasjon av funksjonsfølger}
            \subsubsection*{Integrasjon av funksjonsfølger}
            $\{f_n\}$ er en føge av funksjoner som konvergerer uniformt mot $f$ på [a,b], da er $\lim_{n\to\infty}\int_c^x f_n(t)dt = \int_c^x\lim_{n\to\infty}f_n(t)dt=\int_c^xf(t)dt$
            for $c \in [a,b]$ dette gjelder også for\\ $\lim_{n\to\infty}\int_a^\infty f_n(t)dt = \int_a^\infty\lim_{n\to\infty}f_n(t)dt=\int_c^xf(t)dt$
            \subsubsection*{Derivasjon av funksjonsfølger}
            $\{f_n\}$ er en funksjonsfølge på [a,b], og de deriverte $f'_n$ konvergerer uniformt mot en funksjon h. Anta 
            at $\{f_n(d)\}$ konvergerer for et tall $d\in [a,b]$. Da konvergerer$\{f_n\}$ mot en deriverbar funksjon $f$ og $f'=h$\\
            $\lim_{n\to\infty}f'_n(x)=[\lim_{n\to\infty}f_n(x)]'$
            
    \section*{Rekker}
        \subsection*{Egenskaper ved rekker:}
        La $\sum_{n=0}^\infty a_n$ og $\sum_{n=0}^\infty b_n$ være konvergente rekker:
        \begin{enumerate}[topsep = 0pt,partopsep = 0pt, itemsep = 0cm]
            \item $\sum_{n=0}^\infty (a_n \pm b_n) = \sum_{n=0}^\infty a_n \pm \sum_{n=0}^\infty b_n$
            \item $\sum_{n=0}^\infty ca_n=c\sum_{n=0}^\infty a_n$
        \end{enumerate}
        \subsection*{Absolutt og betinget konvergens}
        \subsubsection*{Definisjon:}
        Vi sier at rekken $\sum a_n$ konvergerer absolutt dersom $\sum\abs{a_n}$ konvergerer.
        \subsubsection*{Lemma:}
        Dersom $\sum a_n$ er betinget konvergent, divergerer både $\sum a_n^+$ og $\sum a_n^-$
        \subsection*{Divergenstesten:}
        $\sum_{n=0}^\infty a_n \text{Konveregerer} \Leftrightarrow \lim_{n\to\infty}a_n=0$
        \subsection*{Integraltesten:}
        Anta $f:[1,\infty)\to\real]$ er en pos., kont. og avtagende funksjon. Da konvergerer rekken $\sum_{n=1}^\infty f(n)$ hviss
        integralet $\int_1^\infty f(x)dx$ \\konveregerer.
        \subsection*{Sammenligningstesten:}
        La $\suminfty{1}{a_n}$ og $\suminfty{1}{b_n}$ være to positive rekker
        \begin{enumerate}
            \item Anta at $\suminfty{1}{a_n}$ konvergerer og at det finnes et tall c slik at $b_n \leq c\cdot a_n$ for alle n. Da konveregerer $\suminfty{1}{b_n}$.
            \item Anta at $\suminfty{1}{a_n}$ divergerer og at det finnes et positivt tall d slik at $b_n \geq d\cdot a_n$ for alle n. Da divergerer $\suminfty{1}{b_n}$
        \end{enumerate}
        \subsection*{Grensesammenligningstesten:}
        La $\suminfty{1}{a_n}$ og $\suminfty{1}{b_n}$ være to positive rekker:
        \begin{enumerate}
            \item Anta at $\suminfty{1}{a_n}$ konvergerer og at $\lim_{n\to\infty}\frac{b_n}{a_n}<\infty$. Da konvergerer også $\suminfty{1}{b_n}$.
            \item Anta at $\suminfty{1}{a_n}$ divergerer og at $\lim_{n\to\infty}>0$. Da \\divergerer også $\suminfty{1}{b_n}$
        \end{enumerate}
        \subsection*{Forholdstesten:}
        La $\suminfty{0}{a_n}$ være en rekke og anta at grensen $\lim_{n\to\infty}\abs{\frac{a_{n+1}}{a_n}}=a$ eksisterer(Den kan være $\infty$!). Da gjelder:
        \begin{enumerate}
            \item Dersom $a<1$, Konveregerer rekken absolutt.
            \item Dersom $a>1$, Divergerer rekken.
            \item Dersom $a=1$, gir testen ingen konklusjon.
        \end{enumerate}
        \subsection*{Rottesten:}
        La $\suminfty{0}{a_n}$ være en rekke og anta at grensen $\lim_{n\to\infty}\sqrt[n]{\abs{a_n}}=a$ eksisterer(den kan være $\infty$!). Da gjelder:
        \begin{enumerate}
            \item Dersom $a<1$, konvergerer rekken absolutt.
            \item Dersom $a>1$, divergerer rekken
            \item Dersom $a=1$, gir testen ingen konklusjon
        \end{enumerate}
        \subsection*{Alternerende rekker test:}
        Anta $\suminfty{1}{a_n}$ er en rekke av typen $\suminfty{1}{(-1)^n b_n}\quad b_n>0$. Hvis;
        \begin{enumerate}
            \item $lim_{n\to\infty}b_n=0$ og
            \item $\{b_n\}$ er en synkende følge
        \end{enumerate}
        Så konvergerer $\suminfty{1}{a_n}$
        \subsection*{Viktige rekker:}
        \begin{enumerate}
            \item Rekken $\sum_{n=1}^\infty \frac{1}{n^p}$ konvergerer hviss $p > 1$ 
        \end{enumerate}
    \section*{Rekker av funksjoner:}
        \subsection*{Weierstrass' M-test:}
        La $\suminfty{0}{v_n(x)}$ være en rekke av funksjoner definert på en mengde A. Anta det finnes en konvergent rekke (av tall) $\sum M_n$ slik at 
        $\abs{v_n(a) \leq M_n}$ for alle n og alle alle $a\in A$. Da konvergerer rekken $\suminfty{0}{v_n(x)}$ uniformt og absolutt på A.
\end{document}